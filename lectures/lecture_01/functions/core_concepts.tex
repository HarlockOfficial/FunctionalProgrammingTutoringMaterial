\section{Functions}
\begin{frame}{Functions}
    % Haskell functions
    \begin{block}{Definition}
        A function is a \textbf{relation} between a set of \textbf{inputs} and a set of \textbf{possible outputs} with the property that each input is related to \textbf{exactly one output}.
    \end{block}
    \begin{block}{Properties}
        \begin{itemize}
            \item \textbf{Pure functions} have no side effects.
            \item \textbf{Higher-order functions} can take functions as arguments and return functions.
            \item \textbf{Recursion} is a technique in which a function calls itself.
        \end{itemize}
    \end{block}
    \begin{block}{Syntax}
        \begin{itemize}
            \item \textbf{Function definition}: \texttt{f x = x + 1}
            \item \textbf{Function application}: \texttt{f 1}
            \item \textbf{Anonymous functions}: \texttt{(\textbackslash x -> x + 1)}
        \end{itemize}
    \end{block}
\end{frame}